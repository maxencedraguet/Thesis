\chapter{\color{oxfordblue} Introduction}
\ChapFrame

Modern particle physics is built around a combined patchwork of theoretical models gathered into the so-called \textit{\gls{sm}} \cite{Schwartz_2013, SMphysics}. In the edifice of science, the \gls{sm} is our current best approach to describing the foundation of Nature at its smallest scale. The elegance of its structure resides in its mingling of mathematical concepts such as symmetries and gauge invariance to model and predict the structure of matter and how it interacts under the strong, the weak, and the electromagnetic interactions. The Brout-Englert-Higgs mechanism plays a central role in the theory, allowing for the emergence of massive gauge vector bosons through spontaneous symmetry breaking \cite{Englert:1964et,  PhysRevLett.13.508}. The success of the theory has been continuosuly confirmed experimentally in countless measurements, particularly with the ATLAS and CMS observation of the Higgs boson in 2012 \cite{ATLAS:2012yve, CMS:2012qbp}. The \gls{sm} is not however a complete theory of the elementary components of the Universe. Gravity is not included, neutrino masses are observed but not accounted for, and the \gls{sm} does not explain astronomical observations of the existence of dark matter. Masses are also introduced arbitrarily for fermions by relying on Yukawa interaction coupling these particles with the Higgs boson. The origin of the strict mass hierarchy between the different generations is therefore unexplained. The \gls{hep} community finds itself in the strange situation of having a remarkably accurante yet incomplete model. Searches are actively ongoing to test all predictions of the \gls{sm}, with the hope to uncover some discrepancies shedding some light on the way forward to correct the theory and include the currently unexplained properties of the Universe. Concerning the Higgs boson, while the leading production modes and the decay modes to third-generation fermions and vector bosons have been observed to agree with the \gls{sm}, the couplings to lighter fermions have not yet been measured. In particular, the coupling to the second-generation $c$-quark can be probed for signs of physics \glsfirst{bsm} \cite{PhysRevD.89.033014,PhysRevD.92.033016,Botella:2016krk,PhysRevD.98.055001,GHOSH2016504,PhysRevLett.123.031802,PhysRevD.100.115041}. \\

This thesis presents, in its last chapter, an ATLAS search for the $H \rightarrow c\bar{c}$ decay mode coupled with a differential cross-section measurement of the $H \rightarrow b\bar{b}$. These analyses are for the first time performed jointly to leverage the same strategy and better constrain the shared backgrounds. The vector boson $V$ ($W$ or $Z$) associated production $VH$ is used, with the $V$ leptonically decaying to 0, 1, or 2 electrons or muons. This latter requirement reduces the otherwise significant multi-jet background and provides an effective signature to select data to save online through the triggers. The search is performed with the 149 fb$^{-1}$ of proton-proton collision data collected by ATLAS during the Run 2 (2015-2018) of the \glsfirst{lhc} at a centre-of-mass energy of 13 TeV. \\

The main ingredient for this analysis consists in reliably identifying $b$- and $c$-quarks from the reconstructed structure of their decay, called a \textit{jet}. This is a challenging task due to the complex structure of jets and the large event rate leading to overlapping signatures. Thankfully, the ATLAS experiment generates a tremendous amount of real and simulated data, leading to the effective deployment of state-of-the-art \glsfirst{ml} techniques. As such, the second central theme of this thesis is the elaboration of evermore complex neural networks called \textit{tagger} to perform jet flavour classification. This thesis extensively describes the latest effort of the ATLAS Collaboration to improve these methods. A complete review is presented on the development and training of DL1d, a hierarchical tagger using the DIPS sub-tagger built on a Deep Set network to replace the DL1r tagger using the RNNIP tagger. A more recent breakthrough in performance has been obtained by adopting a single complex neural network with a powerful \glsfirst{gat} or a Transformer Encoder unit at its core, in models respectively called GN1 and GN2. This thesis presents these new taggers in detail, as well as efforts to leverage techniques from the \glsfirst{ai} community designed for Large Language Models (LLM) to optimise the hyperparameters of the large Transformer-based GN2 network. \\

The thesis aims to present a coherent and connected structure leading to the combined \vhbc\ analysis presented last in Chapter~\ref{chap-VH}. First, the \gls{sm}, the Higgs mechanisms and Yukawa interactions are presented in Chapter~\ref{chap-theory}. The experimental conditions of the \gls{lhc} and the ATLAS experiment are then outlined in Chapter~\ref{chapter-ATLAS}. As machine learning and artificial intelligence play an essential role in modern science, and perhaps even more so in particle physics, Chapter~\ref{Chap-ML} is entirely dedicated to a review of the field relevant to \gls{hep}. Building on this \gls{ml} introduction, Chapter~\ref{chap-ftag} presents the development of the modern flavour taggers of the ATLAS Collaboration, reviewing the design and training of DL1r, DL1d, GN1, and GN2. Finally, the \vhbc\ analysis is described in Chapter~\ref{chap-VH} before concluding with a look forward in Chapter~\ref{chap-Conclusion}.