\section*{Thesis plan}
In my thesis, I will present a coherent and connected story of the research I have carried out during my DPhil. Starting with two chapters introducing respectively the required theoretical backgrounds and the necessary details of the \gls{atlas} experiment, I will then present my work in three main parts.  The thesis will be structured around the following set of chapters:
\begin{description}
\item[Chapter 1] Introduction. \\ \vspace{-15pt}
\item[Chapter 2] Theoretical context: the Standard Model and the Brout-Englert-Higgs mechanism. \\  \vspace{-15pt}
\item[Chapter 3] Experimental context: the Large Hadron Collider and the \gls{atlas} experiment, covering details of both apparatuses, object reconstruction, and other topics relevant to flavour tagging in \gls{atlas}. \\  \vspace{-15pt}
\item[Chapter 4] Flavour tagging: this part will address the challenge of identifying the flavour of jets in the \gls{atlas} experiment and will present my work on the development of an upgraded tagger called \textit{DL1d} compared to what was then the reference, called \textit{DL1r}. For future considerations, I will also introduce and make references to the development of a new graph neural network tagger called \textit{GN1}, as my work on DL1d served as a basis of comparison for the public note, to which I have contributed, introducing this new algorithm to the experiment \cite{ATL-PHYS-PUB-2022-027}. \\  \vspace{-15pt}
\item[Chapter 5] Development of a $Xbb$ tagger: contribution to the development of a $X \rightarrow q\bar{q}$ tagger capable of identifying pairs of $b$- and $c$-jets.
Due to the importance of searches targeting events with a final state made of a $b\bar{b}$ or a $c\bar{c}$ pair, a special task force has been set up in \gls{atlas} at the end of 2022 to deliver a tagger capable of efficiently tagging these states. Such a tagger, called \textit{Xbb}, is naturally of great interest for the $VH (H \rightarrow b\bar{b} / c\bar{c})$ analyses. Our team in Oxford joined the effort in the development of this tagger, which is taking place throughout 2023. The first contribution to the project was the retraining of the DIPS and DL1d algorithm with variable-radius track jets, as this particular training of these two algorithms is required to train the benchmark model to which the new tagger will be compared. The next contribution will be to include neutral information in the new graph neural network being developed, thereby complementing the charged information obtained by analysing the tracks.  \\  \vspace{-15pt}
\item[Chapter 6] This chapter will contain the core of the work carried out during the DPhil on the $VH (H \rightarrow c\bar{c})$ and the $VH (H \rightarrow b\bar{b}/c\bar{c})$ combined analyses using data collected between 2015 and 2018. A coherent and complete overview of the experimental approach used will be presented. Some elements to be discussed are the analysis strategy, studies on the harmonisation of the $VH (H \rightarrow c\bar{c})$ and $VH (H \rightarrow b\bar{b})$, the definition of a common top control region, the full statistical analysis, and, naturally, the outcome of the analysis. \\  \vspace{-15pt}
\item[Chapter 7] Conclusion and outlook.  \\
\end{description}

The objective of the proposed structure is to establish a logical chain connecting the different topics: starting from the theory, moving on to the experimental reality of \gls{atlas}, and then going into the details of flavour tagging. The latter is indeed the most important tool for the $VH (H \rightarrow b\bar{b}/c\bar{c})$ analyses, which will be addressed last as it is the culmination of the work to be presented in the thesis. In the present report, only chapters 4 and 6 as above listed are addressed in further detail, as the work with the $Xbb$ task force is only starting at this point. The $VH (H \rightarrow b\bar{b}/c\bar{c})$ combined analysis is still ongoing at the moment of writing and the status presented here is therefore not final. Due to the complexity of the combined analysis, this report will focus on the $VH (H \rightarrow c\bar{c})$ sub-analysis and, in particular, the study of the background from top-quark decays. In some regions of the analysis, the top process represents up to 80\% of the background at signal-like values of the discriminant variable and is therefore a significant contribution.
