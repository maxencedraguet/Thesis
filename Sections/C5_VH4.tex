\section{Experimental Uncertainties}\label{sec-unc}
Several types of experimental uncertainties need to be taken into account for the combind analysis, to cover systematics effect due to the detector performance, the reconstruction of objects such as leptons and jets, and the effects of flavour tagging. Table \ref{tab:ExpSysts} summarises these various contributions that are sources of uncertainty.

\begin{table}
    \centering
    \renewcommand{\arraystretch}{1.2}
    \resizebox{1.05\textwidth}{!}{%
      \begin{tabular}{llr}
        \hline \hline
        \textbf{Systematic uncertainty name} & $\quad\quad\quad\quad\quad$ \textbf{Description} &\textbf{Regime} \\
        \hline
        \multicolumn{3}{c}{Luminosity and Pile-up}\\
        \hline
        \texttt{LUMI\_2015\_2018 }& Uncertainty on total integrated luminosity & All \\
        \texttt{PRW\_DATASF}& Uncertainty on pile-up modelling & All \\
        \hline
        \multicolumn{3}{c}{\etm and $E_{\text{T,trk}}^{\text{miss}}$}\\
        \hline
        \texttt{MET\_SoftTrk\_ResoPara(Perp)}&  Soft term longitudinal (transverse) resolution uncertainty & All \\
        \texttt{MET\_SoftTrk\_Scale   }&  Soft term scale uncertainty & All \\
        \texttt{MET\_JetTrk\_Scale    }& $E_{\text{T,trk}}^{\text{miss}}$ scale uncertainty & All \\
        \texttt{METTrig\{Stat,Top,Z,Sumpt\}} & Trigger efficiency uncertainty & Resolved \\
        \hline
        \multicolumn{3}{c}{Electrons}\\
        \hline
        \texttt{EL\_EFF\_Trigger\_TOTAL}&  Trigger efficiency uncertainty & All \\
        \texttt{EL\_EFF\_Reco\_TOTAL}&  Reconstruction efficiency uncertainty & All \\
        \texttt{EL\_EFF\_ID\_TOTAL}&  Identification (ID) efficiency uncertainty & All \\
        \texttt{EL\_EFF\_Iso\_TOTAL}&  Isolation efficiency uncertainty & All \\
        \texttt{EG\_SCALE\_ALL}&        Energy scale uncertainty   &  all  \\
        \texttt{EG\_RESOLUTION\_ALL}&    Energy resolution uncertainty  & All \\
        \hline
        \multicolumn{3}{c}{Muons}\\
        \hline
        \texttt{MUON\_EFF\_RECO\_\{STAT,SYS\} }& {Reconstruction and ID efficiency uncertainty for muons with $p_T > 15$ GeV} & All \\
        \texttt{MUON\_EFF\_RECO\_\{STAT,SYS\}\_LOWPT }&{Reconstruction and ID efficiency uncertainty for muons with $p_T \leq 15$ GeV} & All \\
        \texttt{MUON\_EFF\_ISO\_\{STAT,SYS\} }& {Isolation efficiency uncertainty} & All \\
        \texttt{MUON\_EFF\_TTVA\_\{STAT,SYS\} }& {Track-to-vertex association efficiency uncertainty} & All \\
        \texttt{MUON\_SCALE}&    Momentum scale uncertainty & All \\
        \texttt{MUON\_SAGITTA\_RHO(RESBIAS)} & Momentum scale uncertainty to cover charge-dependent local misalignment effects & All \\
        \texttt{MUON\_ID(MS) }& Momentum resolution uncertainty of the inner detector (muon spectrometer) & All \\
        \texttt{MUON\_EFF\_Trig\{Stat,Sys\}Uncertainty} & Trigger efficiency uncertainty & All \\
        \hline
        \multicolumn{3}{c}{Taus}\\
        \hline
        \texttt{TAUS\_TRUEHADTAU\_EFF\_RECO\_TOTAL}& {Reconstruction efficiency} & All \\
        \texttt{TAUS\_TRUEHADTAU\_EFF\_RNNID\_*}& {RNN ID efficiency} & All \\
        \texttt{TAUS\_TRUEHADTAU\_SME\_TES\_*}& {In-Situ tau energy scale correction} & All \\
        \texttt{TAUS\_TRUEELECTRON\_EFF\_ELEBDT\_*}& {Electron Veto efficiency SF} & All \\
        \hline
        \multicolumn{3}{c}{Small-R jets}\\
        \hline
        \texttt{JET\_CR\_BJES\_Response }& Energy scale uncertainties for $b$-jets & All \\
        \texttt{JET\_CR\_EffectiveNP\_Detector\{1-2\} }& Energy scale uncertainties due to in-situ calibration & All \\
        \texttt{JET\_CR\_EffectiveNP\_Mixed\{1-3\} }& Energy scale uncertainties due to in-situ calibration & All  \\
        \texttt{JET\_CR\_EffectiveNP\_Modelling\{1-4\} }& Energy scale uncertainties due to in-situ calibration & All \\
        \texttt{JET\_CR\_EffectiveNP\_Statistical\{1-6\} }& Energy scale uncertainties due to in-situ calibration & All \\
        \texttt{JET\_CR\_EtaIntercal\_Modelling}& Energy scale uncertainties to cover $\eta$-intercalibration non-closure & All \\
        \texttt{JET\_CR\_EtaIntercal\_NonClosure\_highE}& Energy scale uncertainties to cover $\eta$-intercalibration non-closure & All \\
        \texttt{JET\_CR\_EtaIntercal\_NonClosure\_negEta}& Energy scale uncertainties to cover $\eta$-intercalibration non-closure & All \\
        \texttt{JET\_CR\_EtaIntercal\_NonClosure\_posEta}& Energy scale uncertainties to cover $\eta$-intercalibration non-closure & All \\
        \texttt{JET\_CR\_EtaIntercal\_TotalStat}& Energy scale uncertainties to cover $\eta$-intercalibration non-closure & All \\
        \texttt{JET\_CR\_Flav\_Comp(Flavor\_Response) }& Energy scale uncertainty related to flavour composition (response) & All \\
        \texttt{JET\_CR\_PunchTroughMC16 }& Energy scale uncertainty for 'punch-through' & All \\
        \texttt{JET\_CR\_SingleParticle\_HighPt }& Energy scale uncertainty for the behavior of high-\pt\ single hadrons & All \\
        \texttt{JET\_CR\_JER\_DataVsMC}& Energy resolution total uncertainty  & All \\
        \texttt{JET\_CR\_JER\_EffectiveNP\_\{1-6,7restTerm\}}& Energy resolution total uncertainties & All \\
        \texttt{JET\_JvtEfficiency}& JVT efficiency uncertainty  & All \\
        \texttt{JET\_PU\_\{OffsetMu(NPV),PtTerm,RhoTopology\}}& Energy scale uncertainties due to pile-up effects & All \\
        \hline
        \multicolumn{3}{c}{Large-R jets}\\
        \hline
        \texttt{FJ\_JMSJES\_Baseline\_Kin }&  {Energy and mass scale uncertainty due to basic data-simulation differences} & Boosted \\
        \texttt{FJ\_JMSJES\_Modelling\_Kin }& {Energy and mass scale uncertainty due to simulation differences} & Boosted \\
        \texttt{FJ\_JMSJES\_Tracking\_Kin }&  {Energy and mass scale uncertainty on reference tracks} & Boosted \\
        \texttt{FJ\_JMSJES\_TotalStat\_Kin }& {Energy and mass scale uncertainty from stat. unc. on the measurement} & Boosted \\
        \texttt{FJ\_JER }&  Energy resolution uncertainty & Boosted \\
        \texttt{FJ\_JMR }&  Mass resolution uncertainty & Boosted \\
        \hline
        \multicolumn{3}{c}{Flavour tagging: PFlow jets }\\
        \hline
        \texttt{FT\_EFF\_PFlow\_Eigen\_B\_\{0-44\}}& {Tagging efficiency uncertainties for $b$-jets} & Resolved \\
        \texttt{FT\_EFF\_PFlow\_Eigen\_C\_\{0-19\}}&{Tagging efficiency uncertainties for $c$-jets} & Resolved \\
        \texttt{FT\_EFF\_PFlow\_Eigen\_Light\_\{0-19\}}&{Tagging efficiency uncertainties for light-jets} & Resolved \\
        \texttt{FT\_EFF\_PFlow\_extrapolation }& Tagging efficiency uncertainty for high-\pt\ jets & Resolved \\
        \hline
        \multicolumn{3}{c}{$b$-tagging: VR track jets}\\
        \hline
        \texttt{FT\_EFF\_VR\_Eigen\_B\_\{0-4\}}& {$b$-tagging efficiency uncertainties for $b$-jets} & Boosted \\
        \texttt{FT\_EFF\_VR\_Eigen\_C\_\{0-3\}}&{$b$-tagging efficiency uncertainties for $c$-jets} & Boosted \\
        \texttt{FT\_EFF\_VR\_Eigen\_Light\_\{0-3\}}&{$b$-tagging efficiency uncertainties for light-jets} & Boosted \\
        \texttt{FT\_EFF\_VR\_extrapolation }& $b$-tagging efficiency uncertainty for high-\pt\ jets & Boosted \\
        \hline \hline
    \end{tabular}
    }
    \caption{Summary of all experimental systematic uncertainties. }
    \label{tab:ExpSysts}
    \renewcommand{\arraystretch}{1.0}
  \end{table}
   %TODO check that the ftag uncertainties are still completely decorrelated for the VR track and PFlow

\paragraph{Luminosity \& Pile-up:} The measured Run 2 luminosity for ATLAS is 140 fb$^{-1}$ with an uncertainty of 0.83\% \cite{ATLAS:2022hro}. The measurement is performed with $x-y$ beam separation scans combined with information from dedicated lumonisity-sensitive detectors. The pile-up uncertainty for simulated events is obtained by varying the data rescaling factor of the nominal average pile-up $\langle \mu \rangle$. \gls{mc}-simulated samples match data at a higher $\mu$, so this rescaling factor is used to reweight the data, matching a simulated $\mu = 1.0$ to a data-$\mu = 1.09$, written $1.0/1.09$. The 1$\sigma$ uncertainty is measured by varying the factor from $1.0/1.0$ to $1.0/1.18$. % TODO check this is still true: comes from Maria's thesis.

\paragraph{Triggers} Uncertainties on the trigger efficiencies are derived for the electron, muon, and \etm triggers. The electron trigger uncertainty combines the statistical and systematic effect, while for the muon triggers they are considered seperately. Scale factors to the \etm trigger efficiency are derived on $W+$jets events, taking into acount the statistics of the dataset, assessing systematic effects by deriving the \gls{sf}s with alternative top and $Z$+jets samples, and a final uncertainty modelling the efficiency dependency on the scalar sum of all final state jets. % TODO too close to Maria.

\paragraph{Leptons \& \etm\ Reconstruction} Leptons and \etm\ reconstructions were calibrated in dedicated analyses. A reduced set of uncertainties are propagated to the combined \vhbc, and include:
\begin{itemize}
    \item \etm: \gls{sf}s factors are included to account for the direction of the \etm\ as well as the soft term contributions. % TODO link to etm in detector 
    \item Electrons: uncertainties on the reconstructed values, the identification efficiency, isolation efficiency and the energy scale and resolution are included. These are derived by comparing data and simulations kinematic distributions in a $Z \rightarrow e^+ e^-$, $W\rightarrow e\nu$, and $J/\psi \rightarrow e^+e^-$ events \Cite{Aaboud:2657964}. 
    \item Muons: uncertaintes on the reconstruction and identification efficiencies for muons with $p_T > 15$ GeV and $p_T < 15$ are included separately, using respectively samples of $Z\rightarrow \mu^+\mu^-$ and $J/\psi \rightarrow \mu^+\mu^-$ \cite{Aad:2746302}. Additional, uncertainties on the isolation efficiency, track-to-vertex association efficiency, momentum scale and resolution as well as charge-dependent misalignement effects are considered. 
    \item Taus: hadronically decaying $\tau$-leptons\footnote{About $65$\% of $\tau$ decays are hadronic.} uncertainties on the reconstruction and \gls{rnn}-based identification efficiencies as well as the electron veto efficiencies are derived from samples of $Z\rightarrow\tau^+ \tau^-$ and top-quark decays to taus \cite{ATL-PHYS-PUB-2019-033, ATL-PHYS-PUB-2015-045, ATLAS-CONF-2017-029}.
\end{itemize}

\paragraph{Jets} Jets are calibrated in dedicated analyses, of which two reduced sets of uncertainties are propagated to the combined \vhbc for small- and large-$R$ jets. For the small-$R$ jets, these uncertainties cover \textit{in-situ} analyses, $\eta$-intercalibration, flavour composition, punch-through jets, high-$p_T$ hadrons, and pile-up effects as well as the jet energy scale and resolution measured in data \cite{ATLASjesjerMeas, Aad:2854733}. The reduced set is derived by a principal component analysis to preserve correlations in certain regions of jet kinematics. Large-$R$ jets uncertainties for the energy scale and resolution are also estimated from data \cite{ATLAS:2018bip}. An uncertainty covering the calibration discrepancy between data and \gls{mc}-simulations is also included.

\paragraph{Flavour Tagging} A dedicated calibration is performed to derive flavour tagging scale factors in the resolved regime, as described in Section \ref{sec-selectionandcat}, while the common ATLAS uncertainties are used for the boosted regime, as described in \ref{chap-calibration}. These flavour tagging calibration \gls{sf}s are derived by combining data-\gls{mc} efficiency modelling \gls{sf}s and \gls{mc}-\gls{mc} \gls{sf}s to account for parton showering and hadronisation variations. These scale factors are smoothed using a local polynomial kernel estimator to avoid distortions in the kinematic variables \cite{ATL-PHYS-PUB-2020-004}. For each jet flavour, there is one uncertainty per $p_T$ bin in the calibration. A principal component analysis is deployed to reduce the large set of systematic uncertainties to 45 (5) for $b$-jet, 20 (4)for $c$-jet, and 20 (4) for light-jets in the resolved (boosted) regime. An extra uncertainty is added to model the effect of high-$p_T$ jets. Truth tagging uncertainties are covered by these flavour tagging uncertainties.

\section{Signal \& Background Modelling}\label{sec-mod}
The modelling of the signal and background in the \vhbc\ combined analysis is discussed in this section. The signal and background composition changes depending on the lepton channel and the analysis category, with some signal regions background composition represented in Figure \ref{fig:backperchan}. The $V$+jets backgrounds are the dominant ones in the 0-lepton and 2-lepton channels, while the top processes contribute more in the 1-lepton channel and at larger jet multiplicities. Due to flavour tagging, \vhb\ primarily selects the $bb$-component of the background while \vhc\ has more diverse flavour compositions. In summary:
\begin{itemize}
    \item \textbf{0-lepton}: the dominant background is the $Z$+jets with a sizeable $W+$jets component due to large \etm or a miss-identified hadronic $\tau$ and some top backgrounds for the \vhb\ side particularly.
    \item \textbf{1-lepton}: for the \vhb, top processes are dominant while for \vh\ a sizeable $W+$jets is present as well as the top (the latter particularly at higher jet multiplicities). Thre is a visible di-boson contribution as well as some multi-jet.  
    \item \textbf{2-lepton}: the background is mostly $Z+$jets, followed by the di-boson and the top process for \vhb\ (constrained from the Top $e\mu$ CR). 
\end{itemize}

\begin{figure}[h!]
    %\hspace{-2.0cm}
    \center
    \includegraphics[width=0.98\textwidth]{Images/VH/Model/Backperchan/012L.png}
    \caption{Representation of the relative fractions of the different background processes in the 0-lepton (top), 1-lepton (middle), and 2-lepton (bottom) channels for different regimes: resolved \vhb\ (left), \vhc\ (centre), and boosted \vhb\ (right). The regions depicted are the 150 GeV < \ptv\ < 250 GeV 2-jet with $BB$-tag or 2 $c$-tags for the resolved SRs and the 400 GeV < \ptv\ < 600 GeV high-purity boosted SR.}
    \label{fig:backperchan}
\end{figure}
