
%\begin{abstract}
    \vspace*{\fill}
\begin{center}
\textbf{\large \color{oxfordblue} ABSTRACT}
\end{center}
An essential element in many ATLAS analyses is the ability to identify the flavour of jets. This subject is extensively discussed in this thesis, with a review of the algorithmic developments carried out by the ATLAS Collaboration from 2020 to 2024. Increasingly sophisticated Machine Learning (ML) taggers have been developed for this specific purposed. The initial approach relies on a hierarchical construction combining low-level physically-motivated taggers physically with a Deep Set or a Recurrent Neural Network as inputs to a high-level tagger predicting the flavour. Recentely, a more nimble design leveraging a single network to deliver state-of-the-art performance has been introduced. The core of this network is either Graph Attention Network or a Transformer Encoder. Expert knowledge is passed to the architecture by optimising multiple tasks, with different physics input types used in a multimodal framework. The design and training of these taggers are reviewed, with a study of the Hyperparameter Optimisation (HPO) for large networks using techniques from the ML literature on Large Language Models (LLM). \\

Following the 2012 discovery of Higgs boson by the ATLAS and CMS experiments, increasingly refined measurements of the new particles have been performed. The leading production modes and the decay mode to third generation fermions and gauge vector bosons of the Higgs have been measured. Attention is now focused on the second generation fermions, such as the $c$-quark, and differential cross-section measurements. This thesis presents a combined search for the $H \rightarrow c\bar{c}$ coupled with a differential measurement of the $H \rightarrow b\bar{b}$ in the $VH$ production mode. The analysis exploits the full 140 fb$^{-1}$ proton-proton collision luminosity collected in Run 2 by the ATLAS experiment at a centre-of-mass energy of 13 TeV. The combination of these decay modes allow for a coherent joint analysis strategy with a better constraining of the shared backgrounds. The full \pt\ spectrum is covered, with the two candidate jets resolved at low momentum and a single merged boosted signature at high values. The previously reviewed flavour taggers are used to reconstruct the Higgs. Three leptonic channels are defined based on the number of electrons and muons. A fine categorisation is deployed with dedicated Boosted Decision Trees discriminants to increase the sentivity. The analysis is blinded with an expected 95\% CL$_s$ upper limit for the \vhc\ process signal strength at 11.1 $\times$ the SM prediction. The \vhb\ signal strength is 7.9 $\sigma$ over the background-only hypothesis, with the $WH$ and $ZH$ productions respectively measured with expected significances of 5.5 $\sigma$ and 6.2 $\sigma$. A Standard Cross-Section Templates measurement is performed in stage 1.2 for \vhb, in bins of \pt\ and number od additional jets.
%\end{abstract}
\vspace*{\fill}
