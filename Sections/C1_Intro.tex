\chapter{\color{oxfordblue} Introduction}
\ChapFrame

Modern particle physics is built around the \textit{\gls{sm}} \cite{Schwartz_2013, SMphysics}. In the edifice of science, the \gls{sm} is our current best understanding of the foundation of Nature at its smallest scale. The elegance of its structure resides in its implementation of symmetries and gauge invariance to describe and predict the fundamental structure of matter and how it interacts under the strong, the weak, and the electromagnetic interactions. The Brout-Englert-Higgs mechanism plays a central role in the theory, allowing for the emergence of massive gauge vector bosons through spontaneous symmetry breaking \cite{Englert:1964et,  PhysRevLett.13.508}. The \gls{sm} has been experimentally confirmed by countless measurements, particularly with the ATLAS and CMS observation of the theorised Higgs boson in 2012 \cite{ATLAS:2012yve, CMS:2012qbp}. The \gls{sm} is not however a complete theory of the elementary components of the Universe. Gravity is not included, neutrino masses are observed but not accounted for, and the \gls{sm} does not explain astrophysical observations of dark matter. The mass of fermions is introduced somewhat arbitrarily through Yukawa interactions coupling these particles to the Higgs boson. The origin of the mass hierarchy between the different fermionic generations is therefore left unexplained. The \gls{hep} community finds itself in the unsual situation of having a remarkably accurate yet incomplete model. Searches are actively ongoing to test all predictions of the \gls{sm}, with the hope to uncover some discrepancies shedding some light on the way forward to improve the theory and include the currently unexplained properties of the Universe. Concerning the Higgs boson, while the leading production modes and the decay modes to third-generation fermions and vector bosons have been observed to agree with the \gls{sm}, the couplings to lighter fermions have not yet been measured. In particular, the coupling to the second-generation $c$-quark can be probed for signs of physics \glsfirst{bsm} \cite{PhysRevD.89.033014,PhysRevD.92.033016,Botella:2016krk,PhysRevD.98.055001,GHOSH2016504,PhysRevLett.123.031802,PhysRevD.100.115041}. \\

This thesis presents, in its last chapter, an ATLAS search for the $H \rightarrow c\bar{c}$ decay mode and a differential cross-section measurement of the $H \rightarrow b\bar{b}$. These analyses are for the first time performed jointly to better constrain the common backgrounds. The vector boson $V$ ($W$ or $Z$) associated production $VH$ is used, with the $V$ leptonically decaying to 0, 1, or 2 electrons, muons, or neutrinos. This latter requirement reduces the otherwise significant multi-jet background and provides an effective signature to select data to save online through the triggers. The search is performed with the 140 fb$^{-1}$ of proton-proton collision data collected by ATLAS at a centre-of-mass energy of 13 TeV during the Run 2 of the \glsfirst{lhc}, from 2015 to 2018. \\

An essential component in the analysis is to reliably identify $b$- and $c$-quarks from the complex reconstructed structure of their decay called a \textit{jet}. This is a challenging task due to the rich structure of jets and the large event rate. The ATLAS experiment is helped in this mission by the tremendous amount of real and simulated data accessible, leading to the effective deployment of state-of-the-art \glsfirst{ml} techniques. As such, the second central theme of this thesis is the development of sophisticated neural network model called \textit{taggers} to classify the flavour of jets. The recent efforts of the ATLAS Collaboration to improve these methods are extensively described, with a complete account of the algorithmic developments from 2020 to early 2024. First presented is the development and training of DL1d, a hierarchical tagger relying on the DIPS sub-tagger built on a Deep Set network to replace the DL1r tagger using the RNNIP sub-tagger \cite{ATL-PLOT-FTAG-2023-01}. More recentely, a breakthrough in performance has been obtained by adopting a single complex neural network with a powerful \glsfirst{gat} or a transformer encoder unit at its core, in models respectively called GN1 \cite{ATL-PHYS-PUB-2022-027} and GN2 \cite{duperrin2023flavour}. The development, training, and performance of these revolutionary new taggers are reported here in detail, as well as the latest efforts to leverage techniques from the \glsfirst{ai} community designed for Large Language Models (LLM) to optimise the hyperparameters of the large transformer-based GN2 network \cite{yang2021tuning, publicplotMUP}. \\

The thesis strives to present a coherent and connected narrative leading to the combined \vhbc\ analysis presented last in Chapter~\ref{chap-VH}. First, the \gls{sm}, the Higgs mechanism, and Yukawa interactions are reviewed in Chapter~\ref{chap-theory}. The experimental conditions of the \gls{lhc} and the ATLAS experiment are then outlined in Chapter~\ref{chapter-ATLAS}. As machine learning and artificial intelligence play an essential role in modern science, and perhaps even more so in particle physics, Chapter~\ref{Chap-ML} is entirely dedicated to a overview of the field relevant to \gls{hep}. Building on this \gls{ml} introduction, Chapter~\ref{chap-ftag} presents the development of the modern flavour taggers of the ATLAS Collaboration, reviewing the design and training of DL1r, DL1d, GN1, and GN2. Finally, the \vhbc\ analysis is reported in Chapter~\ref{chap-VH} before concluding with a look forward in Chapter~\ref{chap-Conclusion}.