\begin{table}[!htbp]
  \renewcommand*{\arraystretch}{1.3}
  \newcommand\textunderset[2]{\ensuremath{\underset{\text{#1}}{\text{#2}}}}
  \centering
    % resize a bit the table 
    \resizebox{1\textwidth}{!}{%
    \begin{tabular}{c|ccc|ccc}
      \multicolumn{1}{c}{} & \multicolumn{3}{c|}{Resolved \vhb} &  \multicolumn{3}{c}{Boosted \vhb} 
      \\\hline \hline
      \textbf{Settings} & 0L & 1L & 2L & 0L & 1L & 2L
      \\\hline
      Boost type 
        & Gradient boost & Gradient boost & Gradient boost % VHbb Resolved 
        & Adaboost       & Adaboost       & Adaboost       % VHbb Boosted 
      \\\hline 
      Number of trees 
        & 200 & 600 & 200 % VHbb Resolved 
        & 800 & 800 & 400 % VHbb Boosted 
      \\\hline 
      Maximum depth
        & 3 & 4 & 4 % VHbb Resolved 
        & 3 & 3 & 3 % VHbb Boosted 
      \\\hline 
      Learning rate ($\beta$) 
        & 0.5 & 0.5 & 0.5% VHbb Resolved 
        & 0.5 & 0.35 & 0.3 % VHbb Boosted 
      \\\hline 
      Number of cuts
        & 100 & 100 & 100% VHbb Resolved 
        & 60  & 60  & 100 % VHbb Boosted 
      \\\hline 
      Minimum node size 
        & 5\% & 5\% & 5\% % VHbb Resolved 
        & 2\% & 2\% & 7\% % VHbb Boosted 
      \\ \hline \hline
    \end{tabular}}%
  \caption{%
    Hyperparameters of the BDTs in the 0L, 1L and 2L channels for the \vhb resolved and boosted. All models used the Gini index as separation method, without pruning.}%
  \label{tbl:MVAHyperparams}
\end{table}


% VHcc use different hyper parameters
\begin{table}[!htbp]
  \renewcommand*{\arraystretch}{1.3}
  \newcommand\textunderset[2]{\ensuremath{\underset{\text{#1}}{\text{#2}}}}
  \centering
    % resize a bit the table 
    %\resizebox{1\textwidth}{!}{%
    \begin{tabular}{c|cc|c}
      \multicolumn{1}{c}{} & \multicolumn{2}{c|}{\vhc} &  $VZ{\rightarrow c\bar{c}}$ signal
      \\\hline \hline
      \textbf{Settings} & 0L, 1L and most 2L regions & 2  3p-jet, low pTV & 0L, 1L and 2L
      \\\hline
      Boost type 
        & Gradient boost & Adaboost % VHcc
        & Adaboost                  % VZcc
      \\\hline 
      Number of trees 
        & 600 & 200 % VHcc
        & 200       % VZcc 
      \\\hline 
      Maximum depth
        & 4 & 4     % VHcc
        & 4         % VZcc
      \\\hline 
      Learning rate ($\beta$) 
        & 0.5 & 0.15 % VHcc
        & 0.15       % VZcc 
      \\\hline 
      Number of cuts
        & 100 & 100  % VHcc
        & 100        % VZcc
      \\\hline 
      Minimum node size 
        & 5\% & 5\%  % VHcc
        & 5\%        % VZcc
      \\ \hline \hline
    \end{tabular}
  %}
  \caption{%
    Hyperparameters of the BDTs in the 0L, 1L and 2L channels of \vhc. The 2L low pTV region mentioned is 75 GeV $<$ pTV $<$ 150 GeV. All models used the Gini index as separation method, without pruning.}%
  \label{tbl:MVAHyperparams-VHcc}
\end{table}
