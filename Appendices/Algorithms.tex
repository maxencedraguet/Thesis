\chapter{Machine Learning Algorithms}\label{app:MLalgo}
\ChapFrame

\textit{This Appendix presents additional details and pseudocode for selected algorithms presented in Chapter \ref{Chap-ML}.}

\section{Decision Trees and their extensions}
\addtocontents{toc}{\protect\setcounter{tocdepth}{1}}
\subsection{Decision Trees}\label{app:dtalgo}
Algorithm \ref{ag:DT} presents a recursive procedure to train a decision tree. 
\begin{algorithm}
    \caption{Recursive Procedure to Train a Decision Tree \cite{MurphyML}.}
    \begin{algorithmic}
    \Function{fitTree}{node, $D$, depth}
        \State $\text{node.prediction} \gets \text{mean}(\{y_i : i \in $D$\})$ 
        \State $(j^*, t^*, D_L, D_R) \gets \text{split}(D)$
        \If{\text{not worthSplitting}(\text{depth}, \text{cost}, $D_L$, $D_R$)}
            \State \Return node
        \Else
            \State node.left $\gets$ \textsc{FitTree}(node, $D_L$, depth + 1)
            \State node.right $\gets$ \textsc{FitTree}(node, $D_R$, depth + 1)
            \State \Return node
        \EndIf
    \EndFunction
    \end{algorithmic}
    \label{ag:DT}
\end{algorithm}

\subsection{AdaBoost}
Algorithm \ref{algo:adaboost} presents the Adaboost algorithm for binary classification. 
\begin{algorithm}
    \caption{Adaboost for Binary Classification with Exponential Loss \cite{MurphyML}}
    \label{algo:adaboost}
    \begin{algorithmic}
    \State Initialise weights: $w_{i,1} = 1/N$, where $N$ is the number of samples.
    \For{$m = 1$ to $M$}
        \State Minimise $\epsilon_m = \sum_i w_{i,m} \mathbb{I}(y_i \neq f_m(x_i))$ on training set with weights $w_{i,m}$ to find $f_m(x)$.
        \State Compute $\alpha_m = \frac{1}{2} \log\left(\frac{1 - \epsilon_m}{\epsilon_m}\right)$.
        \State Update weights: $w_{i,m+1} \leftarrow w_{i,m} \, \exp(-\alpha_m y_i f_m(x_i))$ and renormalise $\sum_i w_{i, m+1} = 1$.
    \EndFor
    \State \Return $F(x) = \sum_{m=1}^M \alpha_m f_m(x)$
    \end{algorithmic}
\end{algorithm}

\subsection{Gradient Boosting}
Algorithm \ref{algo:gradient_boosting} presents the gradient boosting algorithm. 
\begin{algorithm}
    \caption{Gradient Boosting \cite{MurphyML}}
    \label{algo:gradient_boosting}
    \begin{algorithmic}
    \State Initialise $f_0(x) = \arg\min_\alpha \,\sum_{i=1}^N L(y_i, \alpha)$
    \For{$m = 1$ to $M$}
        \State Compute the gradient residuals for each $i= 1, ..., N$: $g_{i,m} = -\left[\frac{\partial L(y_i, f(x_i))}{\partial f(x_i)}\right]_{f(x_i) = f_{m-1}(x_i)}$
        \State Train weak learner $h_m$ on the dataset $\{(x_i, g_{i,m})\}_{i=1}^N$
        \State Compute $\alpha_m$ by minimising $\sum_{i=1}^N L(y_i, f_{m-1}(x_i) + \alpha h_m(x_i))$
        \State Update $f_m(x) = f_{m-1}(x) + LR \times \alpha_m h_m(x)$
    \EndFor
    \State \Return $f(x) = f_M(x)$
    \end{algorithmic}
\end{algorithm}


\section{Artificial Neural Networks}\label{app:dnn_uni_approx}
\addtocontents{toc}{\protect\setcounter{tocdepth}{1}}
\subsection{Universal Function Approximator Theorem}
A family of theorems establish \glspl{dnn} as Universal Function Approximators. The most famous such theorem states \cite{universalFuncApproxNN,HORNIK1989359}:

\paragraph{Theorem:} \textit{Let $C([0, 1]^n)$ denote the set of all continuous functions $[0, 1]^n \rightarrow \mathbb{R}$ and $\sigma$ be any sigmoidal activation function. Then the finite sum $\hat{f}(x) = \sum_{i=1}^N \alpha_i \sigma(w_i^T x + b_i)$ is dense in $C([0, 1]^n)$. In other words, given any $f \in C([0, 1]^n)$ and $\epsilon > 0$, there is a sum $\hat{f}(x)$ of the above form for which \[ f(x) - \hat{f}(x)| < \epsilon \quad \forall x \in [0, 1]^n.\]}

This essentially states that any well-defined function defined over the $n$-dimensional unit hypercube $[0, 1]^n$ can be approximated by an arbitrarily wide neural network. This result can be applied outside the unit hypercube by a homothetic transformation of the data space. Interestingly, similar theorems were derived for the most important activation functions commonly used in deep learning \cite{universApproximator-Relu}.

\subsection{Backpropagation Algorithm}\label{app:back_pro}
The backpropagation algorithm \ref{ag:backpropagation} is the pinnacle of modern deep learning. Its application in combination with optimised computation graphs allows for the training of large networks on dedicated infrastructure.
\begin{algorithm}
    \caption{Backpropagation Algorithm}
    \begin{algorithmic}
    \Function{Update}{$x$, $y$, $N$, $\mathcal{L}$}
        \State Forward step: propagate input $x$ through network $N$ to get prediction $\hat{y}$
        \State Loss: compute the loss or reward of $N$ as $\mathcal{L}(y, \hat{y})$
        \While{$\exists$ a layer without local gradients}
            \State Take the right-most layer required a gradient
            \State Take the gradient at the input of the subsequent layer
            \State With the chain rule, propagate the gradient of the next layer to the current layer
            \State Store the gradient at the layer
        \EndWhile
    \EndFunction
    \end{algorithmic}
    \label{ag:backpropagation}
\end{algorithm}

\newpage

\subsection{General Graph Update}\label{app:graph_update}
The most general graph update is presented in Algorithm \ref{ag:algorithmic}
\begin{algorithm}
    \caption{Steps of Computation in a Full Graph Network Block \cite{graphInductiveBias}}
    \label{algo:graph_network}
    \begin{algorithmic}
    \Function{GraphNetworkUpdate}{$E, V, u$}
        \For{$k \in \{1 \ldots N^e\}$}
            \State $e^*_k \gets \phi^e(e_k, v_{r_k}, v_{s_k}, u)$
        \EndFor

        \For{$i \in \{1 \ldots N^n\}$}
            \State Let $E_i^* = \{(e^*_k, r_k, s_k)\}$ for $k = 1 : N^e$ where $r_k = i$
            \State $\overline{e^*_i} \gets \rho^{e \to v}(E^*_i)$
            \State $v^*_i \gets \phi^v(\overline{e^*_i}, v_i, u)$
        \EndFor

        \State Let $V^* = \{v^*\}_{i=1}^{N^v}$
        \State Let $E^* = \{(e^*_k, r_k, s_k)\}_{k=1}^{N^e}$
        \State $\overline{e^*} \gets \rho_{e \to u}(E^*)$
        \State $\overline{v^*} \gets \rho_{v \to u}(V^*)$
        \State $u^* \gets \phi_u(\overline{e^*}, \overline{v^*}, u)$

        \State \Return $(E^*, V^*, u^*)$
    \EndFunction
    \end{algorithmic}
    \label{ag:algorithmic}
\end{algorithm}