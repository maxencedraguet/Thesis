%\vspace{-4cm}
\chapter*{\color{oxfordblue} Personal Contributions}

The work presented in this thesis is inherently collaborative, having been carried out as a member of the ATLAS Collaboration. This document focuses on the two subjects to which I mostly contributed: the development of new heavy flavour taggers for the next run of data taking (Run 3) and the combined \vhbc\ analysis using the full Run 2. While I produced some ``ATLAS'' labelled figures, others were taken from public results produced by other members of the Collaboration. Plots without label have been personally produced, with some exceptions in the analysis chapter. This section highlights my personal contributions to these different projects.

\subsubsection{Flavour Tagging}
I joined the flavour tagging group for my qualification task, and contributed to the training of the new taggers. My main contributions are:
\begin{itemize}
    \item Producing training samples with the new ATLAS software release (R22) for Run 3.
    \item Contributions to the Umami software \cite{UmamiCite}: modifications to the preprocessing (introducing importance sampling), handling flexible input variables and output target definition, and modified the postprocessing and results visualisation.
    \item Training the DIPS sub-tagger with \gls{vr} jets \cite{ATL-PHYS-PUB-2020-014}.
    \item First PFlow and \gls{vr} training of DL1d with the DIPS sub-tagger, as well performing a hyperparameter and input variables list optimisation of DL1d \cite{ATL-PLOT-FTAG-2023-01}.
    \item Contributions to the Salt software \cite{SaltCite}: adapting the codebase to the $\mu P$ parametrisation \cite{pmlr-v139-yang21c} and developing a framework to train on CERN's KubeFlow platform \cite{KubeFlowCern}. 
    \item Hyperparameter optimisation studies of GN1 and GN2 with the $\mu P$ parametrisation \cite{publicplotMUP}.
\end{itemize}
These different contributions led me to significantly participate in the development of the \textsc{Umami} \cite{UmamiCite} and \textsc{Salt} \cite{SaltCite} software used to train the networks. My contributions have been part of different ATLAS publications, such as Ref. \cite{ATL-PHYS-PUB-2022-027} with a DL1r model I trained, Ref. \cite{ATL-PLOT-FTAG-2023-01} with a DL1d model I trained, Ref. \cite{ATL-PHYS-PUB-2023-021} for which I produced the DL1d input to the subsequent $X_{bb}$ tagger \cite{ATL-PHYS-PUB-2020-019}, as well as an upcoming ATLAS publication on GN2. I led the effort on the hyperparameter optimisation of GN2, producing the public results in Ref. \cite{publicplotMUP} presented in 2024 at CERN in the $6^{th}$ Inter-Experimental Machine Learning Workshop \cite{publicplotMUP} and the 2024 Cloud-Native AI Day KubeCon conference in Paris \cite{KubeconTalk}. I also presented the recent progress on the development of novel flavour taggers for the Run 3 of ATLAS at the 42$^{\text{nd}}$ International Conference on High Energy Physics (ICHEP) in Prague \cite{Draguet:2906774}.

\subsubsection{\boldvhbc\ Analysis}
I joined the \vhbc\ analysis team in 2021, and my main contributions are:
\begin{itemize}
    \item Comparing the $X_{bb}$ tagger to DL1r for the boosted \vhb\ analysis selection by studying the impact on the signal sensitivity with dedicated MVA trainings.
    \item Contributing to different samples productions and studying the Data-Monte Carlo agreement in \vhc\ with DL1r-based tagging. 
    \item Derivation and harmonisation of the \ptv-dependent $\Delta R_{cc}$ cuts in \vhc.
    \item Designing of a new top control region for the \vhc\ and \vhb\ resolved, with studies leading to the selected approach presented in this thesis. I also investigated different Higgs candidate reconstruction strategy in this control region.
    \item Modelling studies of the top background in the resolved \vhb. Deriving shape and acceptance uncertainties for \ttb, and single-top $Wt$- and $t$-channels, and studying the impact of the chosen modelling and the combination of \ttb\ with $Wt$. Performed the training and deployment of the CARL models for the single-top $Wt$- and $t$-channels of the top background for the resolved \vhb. 
    \item Numerous fit studies to validate new samples, the new top control region and top backgrounds normalisation scheme, and the new Higgs candidate reconstruction strategy as well as studying the impact of the introduction of CARL models and refinements to modelling.
\end{itemize}
At the time of writing, the analysis is reaching its conclusion with final studies on the modelling and the fit framework. It is aiming for publication in the latter part of 2024. The results presented here are therefore not final and the analysis is still blinded.